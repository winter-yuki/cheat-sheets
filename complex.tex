\begin{center}
	\textbf{Комплексный анализ}
\end{center}
\vspace{1ex}

$\displaystyle z = x + iy = r(\cos{\phi} + i\sin{\phi}) = re^{i\phi}$;
$\displaystyle w = f(z) = u(x, y) + iv(x, y) = \rho e^{i\theta}$

\begin{tabular}{|c|c|c|c|c|}
	\hline
	$\displaystyle f'=\cfrac{\partial}{\partial x}(u + iv)$ &
	$\displaystyle \sin{z} = \frac{e^{iz} - e^{-iz}}{2i}$ &
	$\displaystyle \sh{z} = \frac{e^{z} - e^{-z}}{2}$ &
	$\displaystyle \sin{iz} = i\sh{z}$ &
	$\displaystyle \tan{iz} = i\tanh{z}$
	\\

	$\displaystyle f' = \cfrac{\partial}{\partial y}(v -iu)$ &
	$\displaystyle \cos{z} = \frac{e^{iz} + e^{-iz}}{2}$ &
	$\displaystyle \ch{z} = \frac{e^{z} + e^{-z}}{2}$ &
	$\displaystyle \cos{iz} = \ch{z}$ &
	$\cot{iz} = -i\coth{z}$
	\\
	\hline
\end{tabular}
\vspace{1ex}

\noindent
$\displaystyle w = z^\alpha = e^{\alpha Ln(z)} = e^{aln(r) - b(\phi + 2\pi k)}\cdot{e^{i(bln(r)+a\phi+2a\pi k)}}$; \hspace{1em}
$\displaystyle Ln(z) = ln|z| + iArg(z) = ln|z| + i(\phi + 2\pi k)$


\noindent
Стереографическая проекция
$\displaystyle x = \cfrac{2R\xi}{2R-\zeta}$;
$\displaystyle y = \cfrac{2R\eta}{2R-\zeta}$;
$\displaystyle \xi = \cfrac{4R^2x}{x^2+y^2+4R^2}$;
$\displaystyle \eta = \cfrac{4R^2y}{x^2+y^2+4R^2}$;
$\displaystyle \zeta = \cfrac{2R(x^2+y^2)}{x^2+y^2+4R^2}$

\noindent
Условия Коши-Римана $\displaystyle \exists f'(z) \Leftrightarrow$
$\displaystyle \cfrac{\partial u}{\partial x} = \cfrac{\partial v}{\partial y}$,
$\displaystyle \cfrac{\partial u}{\partial y} = -\cfrac{\partial v}{\partial x}$ или
$\displaystyle \cfrac{\partial v}{\partial \phi} = r\cfrac{\partial u}{\partial r}$,
$-\displaystyle \cfrac{\partial u}{\partial \phi} = r\cfrac{\partial v}{\partial r}$ или
$\displaystyle -\rho\cfrac{\partial \theta}{\partial x} = \cfrac{\partial \rho}{\partial y}$,
$\displaystyle \cfrac{\partial \rho}{\partial x} = \rho \cfrac{\partial \theta}{\partial y}$

\noindent
Дробно-линейное преобразование по трем точкам
$\displaystyle \frac{w-w_1}{w-w_2}\frac{w_3-w_2}{w_3-w_1} =
\frac{z-z_1}{z-z_2}\frac{z_3-z_2}{z_3-z_1}$

\vspace{1ex}
\noindent
\begin{tabular}{|c|c|}
	\hline
	$\displaystyle Arcsin(z) = -iLn(i(z + \sqrt{z^2 - 1}))$ &
	$\displaystyle Arctg(z) = \frac{i}{2}Ln\frac{i+z}{i-z} = \frac{1}{2i}Ln\frac{1+iz}{1-iz}$ \\

	$\displaystyle Arccos(z) = -iLn(z + \sqrt{z^2-1})$ &
	$\displaystyle Arcctg(z) = \frac{i}{2}Ln\frac{z-i}{z+i}$ \\

	\hline

	$\displaystyle Arsh(z) = Ln(z+\sqrt{z^2 + 1})$ &
	$\displaystyle Arth(z) = \frac{1}{2}Ln\frac{1+z}{1-z}$ \\

	$\displaystyle Arch(z) = Ln(z+\sqrt{z^2 - 1})$ &
	$\displaystyle Arcth(z) = \frac{1}{2}Ln\frac{z+1}{z-1}$ \\
	\hline
\end{tabular}

\noindent
Гармонические функции
$\displaystyle \Laplace{u}\equiv =
\frac{\partial^2u}{\partial x^2} + \frac{\partial^2u}{\partial y^2} = 0$; \vspace{1em}
$\displaystyle v(x, y) = \int\limits_{(x_0, y_0)}^{(x, y)} \cfrac{\partial v}{\partial x} dx + \cfrac{\partial v}{\partial y} dy + C$



