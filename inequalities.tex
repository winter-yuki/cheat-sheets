\textbf{Неравенства}

Бернулли  \\
$(1+x)^r \ge 1+rx, r\in(-\infty ;0)\cup(1;+\infty )$  \\
$(1+x)^r \le 1+rx, r \in (0;1)$

Неравенство Коши \\
\begin{tabular}{l|l}
    \hline
    \multicolumn{2}{c}{$\bar{x}_{kvadr} \ge \bar{x}_{arithm} \ge \bar{x}_{geom}  \ge \bar{x}_{garmon}$} \\
    $\displaystyle \bar{x}_\mathrm{arithm} = \frac{1}{n} \sum_{i=1}^n{x_i} = \frac{x_1 + x_2 + \ldots + x_n}{n}$ &
    $\displaystyle \bar{x}_\mathrm{geom} = \sqrt[n]{\prod_{i=1}^n{x_i}} = \sqrt[n]{x_1\cdot x_2\cdot\ldots\cdot x_n}$ \\
    $\displaystyle \bar{x}_\mathrm{garmon} = \frac{n}{\sum_{i=1}^n{\frac{1}{x_i}}} = \frac{n}{\frac{1}{x_1} + \frac{1}{x_2} + \ldots + \frac{1}{x_n}}$ &
    $\displaystyle \bar{x}_\mathrm{kvadr} = \sqrt{\frac{1}{n} \sum_{i=1}^n{x_i^2}} = \sqrt{\frac{x_1^2 + x_2^2 + \ldots + x_n^2}{n}}$ \\
    \hline
\end{tabular}
