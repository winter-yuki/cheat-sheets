\documentclass[a4paper,12pt]{article}

% В этом документе преамбула

%%% Работа с русским языком
\usepackage{cmap}					% поиск в PDF
\usepackage{mathtext} 				% русские буквы в формулах
\usepackage[T2A]{fontenc}			% кодировка
\usepackage[utf8]{inputenc}			% кодировка исходного текста
\usepackage[english,russian]{babel}	% локализация и переносы
\usepackage{indentfirst}
\frenchspacing

\renewcommand{\epsilon}{\ensuremath{\varepsilon}}
\renewcommand{\phi}{\ensuremath{\varphi}}
\renewcommand{\kappa}{\ensuremath{\varkappa}}
\renewcommand{\le}{\ensuremath{\leqslant}}
\renewcommand{\leq}{\ensuremath{\leqslant}}
\renewcommand{\ge}{\ensuremath{\geqslant}}
\renewcommand{\geq}{\ensuremath{\geqslant}}
\renewcommand{\emptyset}{\varnothing}
\newcommand*{\ooline}[1]{\overline{\overline{#1}}}
\newcommand*{\defe}{\overset{\underset{\mathrm{def}}{}}{=}}

%%% Дополнительная работа с математикой
\usepackage{amsmath,amsfonts,amssymb,amsthm,mathtools} % AMS
\usepackage{icomma} % "Умная" запятая: $0,2$ --- число, $0, 2$ --- перечисление

%% Номера формул
\mathtoolsset{showonlyrefs=true} % Показывать номера только у тех формул, на которые есть \eqref{} в тексте.
%\usepackage{leqno} % Нумереация формул слева

%% Свои команды
\DeclareMathOperator{\sgn}{\mathop{sgn}}

%% Перенос знаков в формулах (по Львовскому)
\newcommand*{\hm}[1]{#1\nobreak\discretionary{}
{\hbox{$\mathsurround=0pt #1$}}{}}

%%% Работа с картинками
\usepackage{graphicx}  % Для вставки рисунков
\graphicspath{{images/}{images2/}}  % папки с картинками
\setlength\fboxsep{3pt} % Отступ рамки \fbox{} от рисунка
\setlength\fboxrule{1pt} % Толщина линий рамки \fbox{}
\usepackage{wrapfig} % Обтекание рисунков текстом

%%% Работа с таблицами
\usepackage{array,tabularx,tabulary,booktabs} % Дополнительная работа с таблицами
\usepackage{longtable}  % Длинные таблицы
\usepackage{multirow} % Слияние строк в таблице

%%% Теоремы
\theoremstyle{plain} % Это стиль по умолчанию, его можно не переопределять.
\newtheorem{theoremm}{Теорема}[section]
\newtheorem{propositionn}[theoremm]{Утверждение}
\newtheorem{algorithmm}[theoremm]{Алгоритм}
\newenvironment{theorem}[1][Теорема]{%
    \begin{theoremm}[#1]$ $\par\nobreak\ignorespaces
    }{%
    \end{theoremm}
}
\newenvironment{proposition}[1][Утверждение]{%
    \begin{propositionn}[#1]$ $\par\nobreak\ignorespaces
    }{%
    \end{propositionn}
}
\newenvironment{algorithm}[1][Алгоритм]{%
    \begin{algorithmm}[#1]$ $\par\nobreak\ignorespaces
    }{%
    \end{algorithmm}
}

\theoremstyle{definition} % "Определение"
\newtheorem{corollary}{Следствие}[theoremm]
\newtheorem{problem}{Задача}[section]

\theoremstyle{remark} % "Примечание"
\newtheorem*{nonum}{Решение}

%%% Программирование
\usepackage{etoolbox} % логические операторы
\usepackage{listings} 

%%% Страница
\usepackage{extsizes} % Возможность сделать 14-й шрифт
\usepackage{geometry} % Простой способ задавать поля
	\geometry{top=4mm}
	\geometry{bottom=2mm}
	\geometry{left=0mm}
	\geometry{right=1mm}
 %
%\usepackage{fancyhdr} % Колонтитулы
% 	\pagestyle{fancy}
 	%\renewcommand{\headrulewidth}{0pt}  % Толщина линейки, отчеркивающей верхний колонтитул
% 	\lfoot{Нижний левый}
% 	\rfoot{Нижний правый}
% 	\rhead{Верхний правый}
% 	\chead{Верхний в центре}
% 	\lhead{Верхний левый}
%	\cfoot{Нижний в центре} % По умолчанию здесь номер страницы

\usepackage{setspace} % Интерлиньяж
%\onehalfspacing % Интерлиньяж 1.5
%\doublespacing % Интерлиньяж 2
%\singlespacing % Интерлиньяж 1

\usepackage{lastpage} % Узнать, сколько всего страниц в документе.

\usepackage{soul} % Модификаторы начертания

\usepackage{hyperref}
\usepackage[usenames,dvipsnames,svgnames,table,rgb]{xcolor}
\hypersetup{				% Гиперссылки
    unicode=true,           % русские буквы в раздела PDF
    pdftitle={Заголовок},   % Заголовок
    pdfauthor={Автор},      % Автор
    pdfsubject={Тема},      % Тема
    pdfcreator={Создатель}, % Создатель
    pdfproducer={Производитель}, % Производитель
    pdfkeywords={keyword1} {key2} {key3}, % Ключевые слова
    colorlinks=true,       	% false: ссылки в рамках; true: цветные ссылки
    linkcolor=red,          % внутренние ссылки
    citecolor=black,        % на библиографию
    filecolor=magenta,      % на файлы
    urlcolor=cyan           % на URL
}

\usepackage{csquotes} % Еще инструменты для ссылок

%\usepackage[style=authoryear,maxcitenames=2,backend=biber,sorting=nty]{biblatex}

\usepackage{multicol} % Несколько колонок

\usepackage{tikz} % Работа с графикой
\usepackage{pgfplots}
\usepackage{pgfplotstable}


\usepackage{listings}
\usepackage{color}
\definecolor{mygreen}{rgb}{0,0.6,0}
\definecolor{mygray}{rgb}{0.5,0.5,0.5}
\definecolor{mymauve}{rgb}{0.58,0,0.82}
\lstset{ 
    backgroundcolor=\color{white},   % choose the background color; you must add \usepackage{color} or \usepackage{xcolor}; should come as last argument
    basicstyle=\footnotesize\ttfamily,        % the size of the fonts that are used for the code
    breakatwhitespace=false,         % sets if automatic breaks should only happen at whitespace
    breaklines=true,                 % sets automatic line breaking
    captionpos=b,                    % sets the caption-position to bottom
    commentstyle=\color{mygreen},    % comment style
    deletekeywords={...},            % if you want to delete keywords from the given language
    escapeinside={\%*}{*)},          % if you want to add LaTeX within your code
    extendedchars=true,              % lets you use non-ASCII characters; for 8-bits encodings only, does not work with UTF-8
    keepspaces=true,                 % keeps spaces in text, useful for keeping indentation of code (possibly needs columns=flexible)
    keywordstyle=\color{blue},       % keyword style
    language=C++,                 % the language of the code
    morekeywords={*,...},            % if you want to add more keywords to the set
    numbers=left,                    % where to put the line-numbers; possible values are (none, left, right)
    numbersep=10pt,                   % how far the line-numbers are from the code
    numberstyle=\color{mygray}, % the style that is used for the line-numbers
    rulecolor=\color{black},         % if not set, the frame-color may be changed on line-breaks within not-black text (e.g. comments (green here))
    showspaces=false,                % show spaces everywhere adding particular underscores; it overrides 'showstringspaces'
    showstringspaces=false,          % underline spaces within strings only
    showtabs=false,                  % show tabs within strings adding particular underscores
    stringstyle=\color{mymauve},     % string literal style
    tabsize=2,	                   % sets default tabsize to 2 spaces
}




%\includeonly{chapters/ch2,chapters/ch3}

\begin{document}


\begin{tabular}{l|l|l}
    \multicolumn{3}{c}{Тригонометрические функции} \\
    \hline
    
    $\displaystyle sin^2(x) + cos^2(x) = 1$  & 
    $\displaystyle tg^2(x) + 1 = \frac{1}{cos^2(x)}$ &
    $\displaystyle tg(x) = \frac{sin(x)}{cos(x)}$ \\
    
    $\displaystyle tg(x)ctg(x) = 1$ &
    $\displaystyle ctg^2(x) + 1 = \frac{1}{sin^2(x)}$ &
    $\displaystyle ctg(x) = \frac{cos(x)}{x}$ \\  
    \hline
    
    $\displaystyle sin(x \pm y) = sin(x)cos(y) \pm cos(x)sin(y)$ & 
    $\displaystyle sin(x) = \frac{2tg\left(\frac{x}{2}\right)}{1 + tg^2\left(\frac{x}{2}\right)}$ &
    $\displaystyle sin(2x) = 2sin(x)cos(x)$ \\
    
    $\displaystyle cos(x \pm y) = cos(x)cos(y) \mp sin(x)sin(y)$ & 
    $\displaystyle cos(x) = \frac{1 - tg^2\left(\frac{x}{2}\right)}{1 + tg^2\left(\frac{x}{2}\right)}$ &
    $\displaystyle cos(2x) = cos^2(x) - sin^2(x)$ \\
    
    $\displaystyle tg(x \pm y) = \frac{tg(x) \pm tg(y)}{1 \mp tg(x)tg(y)}$ & 
    $\displaystyle tg(x) = \frac{2tg\left(\frac{x}{2}\right)}{1 - tg^2\left(\frac{x}{2}\right)}$ &
    $\displaystyle tg(2x) = \frac{2tg(x)}{1-tg^2(x)}$ \\
    
    $\displaystyle ctg(x \pm y) = \frac{-1 \pm ctg(x)ctg(y)}{ctg(x) \pm ctg(y)}$ & 
    $\displaystyle ctg(x) = \frac{1 - tg^2\left(\frac{x}{2}\right)}{2tg\left(\frac{x}{2}\right)}$ &
    $\displaystyle ctg(2x) = \frac{ctg^2(x) - 1}{2ctg(x)}$ \\
    \hline
    
    $\displaystyle sin(x) \pm sin(y) = 2sin\left(\frac{x \pm  y}{2}\right)cos\left(\frac{x \pm y}{2}\right)$ &
    $\displaystyle sin\left(\frac{x}{2}\right) = \sqrt{\frac{1 - cos(x)}{2}}$ &
    $\displaystyle sin^2(x) = \frac{1 - cos(2x)}{2}$ \\ 
    
    $\displaystyle cos(x) + cos(y) = 2cos\left(\frac{x +  y}{2}\right)cos\left(\frac{x - y}{2}\right)$ &
    $\displaystyle cos\left(\frac{x}{2}\right) = \sqrt{\frac{1 + cos(x)}{2}}$ &
    $\displaystyle cos^2(x) = \frac{1 + cos(2x)}{2}$ \\  
    
    $\displaystyle sin(x) \pm sin(y) = 2sin\left(\frac{x \pm  y}{2}\right)cos\left(\frac{x \pm y}{2}\right)$ &
    $\displaystyle tg\left(\frac{x}{2}\right) = \sqrt{\frac{1 - cos(x)}{1 + cos(x)}}$ &
    $\displaystyle (sin(x) + cos(x))^2 = 1 + sin(2x)$ \\  
    \hline
    
    $\displaystyle sin(x)sin(y) = \frac{1}{2}(cos(x - y) - cos(x + y))$ &
    $\displaystyle $ &
    $\displaystyle $ \\  
    
    $\displaystyle cos(x)cos(y) = \frac{1}{2}(cos(x - y) + cos(x + y))$ &
    $\displaystyle $ &
    $\displaystyle $ \\  
    
    $\displaystyle sin(x)cos(y) = \frac{1}{2}(sin(x - y) + sin(x + y))$ &
    $\displaystyle $ &
    $\displaystyle $ \\  
    \hline
    
    \multicolumn{3}{c}{Гиперболические функции} \\
    \hline
    
    $\displaystyle \operatorname{sh}x=\frac{e^x-e^{-x}}{2}$ &
    $\displaystyle \operatorname{sh}(x \pm y)=\operatorname{sh}x\,\operatorname{ch}y \pm \operatorname{sh}y\,\operatorname{ch}x.$ &
    $\displaystyle \operatorname{ch}^2\frac{x}{2} = \frac{\operatorname{ch} x + 1}{2}.$ \\  
    
    $\displaystyle \operatorname{ch}x=\frac{e^x+e^{-x}}{2}$ &
    $\displaystyle \operatorname{ch}(x \pm y)=\operatorname{ch}x\,\operatorname{ch}y \pm \operatorname{sh}y\,\operatorname{sh}x.$ &
    $\displaystyle \operatorname{sh}^2\frac{x}{2} = \frac{\operatorname{ch} x - 1}{2}.$ \\  
    
    $\displaystyle \operatorname{ch}^2t-\operatorname{sh}^2t=1$ &
    $\displaystyle \operatorname{th}(x \pm y)=\frac{\operatorname{th}x \pm \operatorname{th}y}{1 \pm \operatorname{th}x\,\operatorname{th}y}.$ &
    $\displaystyle $ \\  
    
    $\displaystyle $ &
    $\displaystyle \operatorname{cth}(x \pm y)=\frac{ 1 \pm \operatorname{cth}x\,\operatorname{cth}y}{\operatorname{cth}x \pm \operatorname{cth}y}.$ &
    $\displaystyle $ \\  
    \hline
    
%    $\displaystyle $ &
%    $\displaystyle $ &
%    $\displaystyle $ \\  
%    
%    $\displaystyle $ &
%    $\displaystyle $ &
%    $\displaystyle $ \\  
%    
%    $\displaystyle $ &
%    $\displaystyle $ &
%    $\displaystyle $ \\  
%    
%    $\displaystyle $ &
%    $\displaystyle $ &
%    $\displaystyle $ \\    
\end{tabular}

\vspace{2ex}
При $x \to 0$
\begin{equation}
    \begin{aligned}
        &sin(x) \approx x \\
        &tg(x) \approx x \\
        &ln(1+x) \approx x \\
        &a^x - 1 \approx xln(a) \\
        &e^x - 1 \approx x \\
        &(1 + x)^\alpha - 1 \approx \alpha x \\
        &arcsin(x) \approx x \\
        &arctg(x) \approx x \\
    \end{aligned}
\end{equation}



\end{document} % конец документа

