\begin{center}
    \textbf{Другие интегралы}
\end{center}

\textit{Поверхностный интеграл первого рода}

$\displaystyle \iint\limits_{(S)}f(x, y, z)dS = \iint\limits_{(D)}f(x, y, z(x, y))\sqrt{1+(z_x')^2+(z_y')^2}dxdy = \iint\limits_{(\Delta)}f(x(u, v), y(u, v), z(u, v))\sqrt{EG - F^2}dudv$

$\displaystyle \vec{\tau_u} = (x_u',y_u',z_u'); ~\vec{\tau_v} = (x_v',y_v',z_v'); ~E = ||\vec{\tau_u}||^2; ~G = ||\vec{\tau_v}||^2; ~F = \vec{\tau_u} \cdot \vec{\tau_v}$

\vspace{1ex}
\textit{Поверхностный интеграл второго рода}

$\displaystyle \iint\limits_{(S)}\vec{F}d\vec{S} = \iint\limits_{(S)}Pdydz + Qdzdx + Rdxdy = \iint\limits_{(S)}\vec{F}\cdot\vec{n}_0~dS = \iint\limits_{(S)}(P\cos\alpha+\ldots)dS = \pm\iint\limits_{(D)}P(x,y,f(x,y))dxdy+\ldots$

\vspace{1ex}
\textit{Обобщённая формула Стокса:} 
$\displaystyle\int\limits_{\partial{S}}w = \int\limits_{S}dw$;
\hspace{1ex}$\partial{S}$ --- граница области S.
\vspace{-2.5ex}
\begin{itemize}
    \setlength\itemsep{-0.5em}
    \item Формула Грина: $\displaystyle\oint\limits_{(K)}P(x, y)dx + Q(x, y)dy = \iint\limits_{(D)}\left(\frac{\partial Q}{\partial x} - \frac{\partial P}{\partial y}\right)dxdy$
    \item Формула Стокса: $\displaystyle \oint\limits_{(K)} Pdx + Qdy + Rdz = \iint\limits_{(S)}\begin{vmatrix}
    \cos{\alpha} & \cos{\beta} & \cos{\gamma} \\
    \frac{\partial}{\partial{x}} & \frac{\partial}{\partial{y}} & \frac{\partial}{\partial{z}} \\
    P & Q & R \\
    \end{vmatrix}dS =  \iint\limits_{(S)} \operatorname{rot}\vec{F} \vec{dS}$ 
    \item Формула Гаусса-Остроградского: $\displaystyle  \iint\limits_{(S)} \vec{F}\vec{dS} = \iiint\limits_{(V)} \operatorname{div}\vec{F}dxdydz$ 
\end{itemize}

$\displaystyle V = \iiint\limits_{(T)}dxdydz = \frac{1}{3}\iint\limits_{(S)}xdydz+ydxdz+zdxdy = \frac{1}{3}\iint\limits_{(S)}(x\cos{\alpha} + y\cos{\beta} + z\cos{\gamma})dS$























