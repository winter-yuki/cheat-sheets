
$M_{\text{э}}$ --- энерг. светимость / интегральная излучающая способность - плотность потока энергии, испускаемой по всем направлениям.

$u_{\omega,T}$ --- спектральная испуск. способность.
\[u_{\omega,T} \rightarrow d\omega; u_{\lambda, T} \rightarrow d\lambda\]

\[M_{\text{э}} = \int_{0}^{\infty}u_{\omega, T}d\omega \]

1). Закон Стефана-Больцмана: $M_{\text{э}} = \sigma T^4; \sigma = const$.

2). Закон смещения Вина: $\lambda_{max} = \frac{b}{T}; b = const$.

$u$ - объёмная плотность энергии (без индексов).

$M_{\text{э}} = \cfrac{cu}{4}; c\text{ --- скорость света в вакууме}$.

$p = \cfrac{u}{3}; p\text{ --- давление теплового излучения}$.

$\alpha_{\omega, T}$ --- поглощательная способность.

Закон Кирхгофа:
$\cfrac{u_{\omega,T}}{\alpha_{\omega,T}} = f(\omega,T)$ --- универсальная функция Кирхгофа.

Абсолютно чёрное тело: $\alpha_{\omega,T} = 1 \Rightarrow u_{\omega,T} = f(\omega, T); u_{\omega} = \omega^3F(\frac{\omega}{T})$.

$E = M_{\text{э}}S\Delta t = mc^2; u_{\omega, T}\cfrac{2\pi c}{\lambda^2}d\lambda = u(\lambda, T)d\lambda$.

$E = \hbar \omega = h \nu; p = \hbar k = \frac{2\pi}{\lambda}\hbar$

$T + R + A = 1$ --- поглощ. + отраж. + пропуск.

Формула Планка:
\[
u_{\omega} = \cfrac{\hbar \omega^3}{\pi^2 c^3} \times \cfrac{1}{e^cfrac{\hbar \omega}{kT} - 1}.
\]

\[
\hbar \omega \ll kT \Rightarrow u_{\omega} = \cfrac{\omega^2}{\pi^2c^3}kT
\]

\[
\hbar \omega \gg kT \Rightarrow u_{\omega} = \cfrac{\hbar \omega^3}{\pi^2 c^3}e^{-\cfrac{\hbar\omega}{kT}}
\]

$E = IS\Delta t \cos(\vartheta); F = \cfrac{dp}{dt} = \cfrac{E}{c}; T = eU$

1). Тормозное: $\hbar \omega_0 \le eU$.

2). Характеристическое: $h\nu = R(z - \sigma)^2(\cfrac{T}{m^2}-\cfrac{1}{n^2})$

$ R\text{ --- постоянная Р.}; z\text{ --- пор. номер эл-та}; \sigma \text{ --- параметр экранирования}$.

\[
T = mc^2\left(
\cfrac{1}{\sqrt{1-\beta^2}-1}
\right) ; (\beta = \frac{v}{c})
\]

$\sigma_0$ --- темновая элетктропров-сть; $\sigma'$ --- при освещ.

$\Delta\sigma = \sigma' - \sigma_0$ --- фотопров.

$\sigma = e n \mu; n$ --- концентрация; $\mu = \frac{v}{E}; n = n_0 + \Delta n$

$\omega = 2\pi\frac{c}{\lambda}; \lambda = \frac{c}{\nu}$

$\Delta \lambda = \lambda' - \lambda = \lambda_c(1 - \cos(\vartheta))$

$\omega_0^2 = \frac{k}{m}; eU_k = A_1 - A_2$

$E_{\text{п.}} = \cfrac{mv^2}{2} + \cfrac{kx^2}{2}$

$mvr = n\hbar; \cfrac{mv^2}{r} = \cfrac{kze^2}{r^2}; U = -\cfrac{kze^2}{r}; R = \cfrac{kme^4}{2\hbar^2}$

Переходы: Лаймана --- на первый уровень; Бальмера --- на второй уровень.

$\oint \overline{p} d \overline{q} = 2\pi n\hbar; E = \cfrac{p^2}{2m}; U \sim x^2$ --- осциллятор.

$\Delta x \Delta p_x \ge \hbar; \Delta E \Delta t \ge \hbar$

Уравнение Шрёдингера:
\[
\cfrac{d^2\psi}{dx^2} + \cfrac{2m}{\hbar^2}\left(E - U\right)\psi = 0
\]

\[
-\cfrac{\hbar^2}{2m} \cfrac{d^2\psi}{dx^2} = E\psi; \psi = A \sin \left(\cfrac{x\sqrt{2mE})}{\hbar}\right) + B \cos \left(
\cfrac{x\sqrt{2mE}}{\hbar}\right)
\]

$E_n = \cfrac{n^2\hbar^2\pi^2}{2ml^2}$

$A: |\psi|^2 = \psi \times \overline{\psi}; \int_{0}^{l}|\psi|^2dx=1$

$A = \sqrt{\cfrac{2n}{l}}; <x> = \int_{-\infty}^{\infty}x|\psi|^2dx$

$<p_x> = <\overline{\psi} | p_x^{\wedge} | \psi>; \overline{p}^{\wedge} = -i\hbar \nabla \Rightarrow p_x^{\wedge} = -i\hbar \frac{\partial}{\partial x}$

$E = <U>$