\begin{center}
    \textbf{Классическая вероятность}
\end{center}

$A + B = A + \bar{A}B$;
$\displaystyle \overline{\sum_{i=1}^{n}A_i} = \prod_{i=1}^{n}\bar{A}_i$;
$\displaystyle \sum_{i=1}^{n}\bar{A}_i = \overline{\prod_{i=1}^{n}A_i}$

Условная вероятность
$\displaystyle P(B|A) = \frac{P(AB)}{P(A)}$;
$\displaystyle P\left(\prod_{k=1}^{n}A_k\right) = P(A_1)\cdot P(A_2|A_1)\cdot\ldots\cdot P\left(A_n\left|\right.\prod_{k=1}^{n-1}A_k\right)$

События независимы в совокупности
$P(\prod A_k) = \prod{P(A_k)}$;
$P(\sum{A_k}) = 1-\prod[1-P(A_k)]$

Попарно несовместны $P(\sum{A_k}) = \sum{P(A_k)}$

$P(A) = \sum P(H_i)P(A|H_i)$;
$\displaystyle P(H_k|A) = \frac{P(H_k)P(A|H_k)}{\sum P(H_i)P(A|H_i)}$

Бернулли
$P^{(n)}_m = \binom{n}{m}p^mq^{n-m}$;
Пуасона
$\displaystyle P_m = \frac{a^m}{m!}e^{-a}$;
полиномиальный
$P^{(n)}_{m_1\ldots m_k} = \binom{n}{m_1\ldots m_k}p_1^{m_1}\ldots p_k^{m_k}$
